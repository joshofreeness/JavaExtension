\documentclass[twocolumn]{article} 	% use "amsart" instead of "article" for AMSLaTeX format
\usepackage{geometry}                		% See geometry.pdf to learn the layout options. There are lots.
\geometry{letterpaper}                   		% ... or a4paper or a5paper or ... 
%\geometry{landscape}                		% Activate for for rotated page geometry
%\usepackage[parfill]{parskip}    		% Activate to begin paragraphs with an empty line rather than an indent
\usepackage{graphicx}				% Use pdf, png, jpg, or eps§ with pdflatex; use eps in DVI mode
								% TeX will automatically convert eps --> pdf in pdflatex		
\usepackage{amssymb}
\usepackage{titling}

\usepackage{color}
\usepackage{listings}
\lstset{ %
language=Java,                % choose the language of the code
basicstyle=\footnotesize,       % the size of the fonts that are used for the code
numbers=left,                   % where to put the line-numbers
numberstyle=\tiny,      % the size of the fonts that are used for the line-numbers
stepnumber=1,                   % the step between two line-numbers. If it is 1 each line will be numbered
numbersep=4pt,                  % how far the line-numbers are from the code
backgroundcolor=\color{white},  % choose the background color. You must add \usepackage{color}
showspaces=false,               % show spaces adding particular underscores
showstringspaces=false,         % underline spaces within strings
showtabs=false,                 % show tabs within strings adding particular underscores
frame=single,           % adds a frame around the code
tabsize=2,          % sets default tabsize to 2 spaces
captionpos=b,           % sets the caption-position to bottom
breaklines=true,        % sets automatic line breaking
breakatwhitespace=false,    % sets if automatic breaks should only happen at whitespace
escapeinside={\%*}{*)}          % if you want to add a comment within your code
}

% Add your keywords here, 
% and include it in your preamble
\lstset{emph={%  
    extendsAll,%
    },emphstyle={\bfseries}%
}%


\title{Java Extension  \\ \textsc{\small Multiple Inheritance}}
%\subtitle{Multiple Inheritance}
\author{Joshua Free}
\date{}							% Activate to display a given date or no date

\begin{document}
\maketitle
\section{Java Extension}
This chapter discusses the pros and cons of multiple inheritance, the specifics of my implementation and the extra semantics that must be added to support such a feature.

\subsection{Multiple Inheritance}

Multiple Inheritance is implemented in many languages including well known languages such as C++, Python, Perl and Scala. Languages that do not support multiple inheritance are Java, C\# and Ruby. These languages provide similar but not identical functionality though interfaces and similar constructs.% \cite {against}


Ever since multiple inheritance was introduced in 1991 in C++ 2.0 \cite{date} there have been debates concerning the pros and cons of the approach. A common argument against multiple inheritance is that there are very little or no examples of cases where multiple inheritance is required. In an article Waldo states that this may be due to the nature of multiple inheritance as it lends itself more to large systems than small and understandable examples. \cite{examples}  %Can make these reference pages with \cite[p. 5]{} if needed


A common concern when interacting with multiple inheritance is ``The diamond problem". This describes the issue of a child class not knowing which method or field to access if two parents at the same level in the inheritance hierarchy have fields or methods named the same. This is often due to the two parent classes themselves extending a single class and overriding its methods. In this extension I don't allow such a case by checking if parent classes containing methods or fields with the same name.


A benefit to multiple inheritance is code reuse. In an interface approach if multiple classes were to implement a single interface then they need code duplication to satisfy the interface. However in coding languages with multiple inheritance all classes that inherit from the classes all get the implementation without having to repeat code



\subsection{Feature}
This feature works similar to other languages. A single class is able to extend multiple classes. 
\begin{lstlisting}
public class A extendsAll B, C {	
	.....
	}
	
public class A extendsAll B, C, D, E, F {
	.....
	}	
\end{lstlisting}
As you can see the keyword extendsAll is used to differentiate between the new feature and standard Java code. The feature can allow a user to make a class extend any number of classes. Due to the nature of Java only one class in the list of parent classes may extend another class. %\cite {for}

\subsection{Semantics}
To support this feature in Java some semantic analysis had to be introduced. Most prominently relating to the diamond issue. In order to avoid this error the parent classes in an inheritance tree are scanned and analysed to make sure that they do not contain fields or methods that conflict with one another. If this is observed the compiler will throw an exception.

In order to create a functioning Java class hierarchy only one class being extended may implement another class. The semantics of this is checked in my compiler and an error is thrown when too many class extend other classes.

While implementing my feature I mainly dealt with the ClassOrInterfaceDeclaration node in the AST tree. This node represents a declaration of an interface or a class and has a list of ClassOrInterfaceTypes which are the classes which it extends. Once these nodes are obtained analysis can be performed to first match up ClassOrInterfaceDeclaration nodes with their associated ClassOrInterfaceTypes. This step makes discernible whether the classes to be extended extend any other classes and how many.

The class which is allowed to extend another class is placed on the top of a linear class hierachy and the rest extend the class in a chain, with the ``extendAll" class placed at the bottom.

Due to the direct manipulation of the ClassOrInterfaceDeclaration nodes from ``javax" code to  correct java code, I was able to produce my source java code with no modification to the dump visitor.
\section{Semantics}
This section will discuss some of the semantics implemented that extend further than the feature.
\subsection{Tokens}
In a jj file many tokens can be defined, in order to implement this feature I had to compliment the existing tokens with the new \emph{extendAll}  token. After defining this token I could attach information to the ClassOrInterfaceDeclaration node for later reference.
\subsection{Expressions}
Expressions are a major part of any language and the time spent on expression semantics reflects this. 



\bibliography{report}
\bibliographystyle{ieeetran}

\end{document}  